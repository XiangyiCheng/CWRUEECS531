
% Default to the notebook output style

    


% Inherit from the specified cell style.




    
\documentclass[11pt]{article}

    
    
    \usepackage[T1]{fontenc}
    % Nicer default font (+ math font) than Computer Modern for most use cases
    \usepackage{mathpazo}

    % Basic figure setup, for now with no caption control since it's done
    % automatically by Pandoc (which extracts ![](path) syntax from Markdown).
    \usepackage{graphicx}
    % We will generate all images so they have a width \maxwidth. This means
    % that they will get their normal width if they fit onto the page, but
    % are scaled down if they would overflow the margins.
    \makeatletter
    \def\maxwidth{\ifdim\Gin@nat@width>\linewidth\linewidth
    \else\Gin@nat@width\fi}
    \makeatother
    \let\Oldincludegraphics\includegraphics
    % Set max figure width to be 80% of text width, for now hardcoded.
    \renewcommand{\includegraphics}[1]{\Oldincludegraphics[width=.8\maxwidth]{#1}}
    % Ensure that by default, figures have no caption (until we provide a
    % proper Figure object with a Caption API and a way to capture that
    % in the conversion process - todo).
    \usepackage{caption}
    \DeclareCaptionLabelFormat{nolabel}{}
    \captionsetup{labelformat=nolabel}

    \usepackage{adjustbox} % Used to constrain images to a maximum size 
    \usepackage{xcolor} % Allow colors to be defined
    \usepackage{enumerate} % Needed for markdown enumerations to work
    \usepackage{geometry} % Used to adjust the document margins
    \usepackage{amsmath} % Equations
    \usepackage{amssymb} % Equations
    \usepackage{textcomp} % defines textquotesingle
    % Hack from http://tex.stackexchange.com/a/47451/13684:
    \AtBeginDocument{%
        \def\PYZsq{\textquotesingle}% Upright quotes in Pygmentized code
    }
    \usepackage{upquote} % Upright quotes for verbatim code
    \usepackage{eurosym} % defines \euro
    \usepackage[mathletters]{ucs} % Extended unicode (utf-8) support
    \usepackage[utf8x]{inputenc} % Allow utf-8 characters in the tex document
    \usepackage{fancyvrb} % verbatim replacement that allows latex
    \usepackage{grffile} % extends the file name processing of package graphics 
                         % to support a larger range 
    % The hyperref package gives us a pdf with properly built
    % internal navigation ('pdf bookmarks' for the table of contents,
    % internal cross-reference links, web links for URLs, etc.)
    \usepackage{hyperref}
    \usepackage{longtable} % longtable support required by pandoc >1.10
    \usepackage{booktabs}  % table support for pandoc > 1.12.2
    \usepackage[inline]{enumitem} % IRkernel/repr support (it uses the enumerate* environment)
    \usepackage[normalem]{ulem} % ulem is needed to support strikethroughs (\sout)
                                % normalem makes italics be italics, not underlines
    

    
    
    % Colors for the hyperref package
    \definecolor{urlcolor}{rgb}{0,.145,.698}
    \definecolor{linkcolor}{rgb}{.71,0.21,0.01}
    \definecolor{citecolor}{rgb}{.12,.54,.11}

    % ANSI colors
    \definecolor{ansi-black}{HTML}{3E424D}
    \definecolor{ansi-black-intense}{HTML}{282C36}
    \definecolor{ansi-red}{HTML}{E75C58}
    \definecolor{ansi-red-intense}{HTML}{B22B31}
    \definecolor{ansi-green}{HTML}{00A250}
    \definecolor{ansi-green-intense}{HTML}{007427}
    \definecolor{ansi-yellow}{HTML}{DDB62B}
    \definecolor{ansi-yellow-intense}{HTML}{B27D12}
    \definecolor{ansi-blue}{HTML}{208FFB}
    \definecolor{ansi-blue-intense}{HTML}{0065CA}
    \definecolor{ansi-magenta}{HTML}{D160C4}
    \definecolor{ansi-magenta-intense}{HTML}{A03196}
    \definecolor{ansi-cyan}{HTML}{60C6C8}
    \definecolor{ansi-cyan-intense}{HTML}{258F8F}
    \definecolor{ansi-white}{HTML}{C5C1B4}
    \definecolor{ansi-white-intense}{HTML}{A1A6B2}

    % commands and environments needed by pandoc snippets
    % extracted from the output of `pandoc -s`
    \providecommand{\tightlist}{%
      \setlength{\itemsep}{0pt}\setlength{\parskip}{0pt}}
    \DefineVerbatimEnvironment{Highlighting}{Verbatim}{commandchars=\\\{\}}
    % Add ',fontsize=\small' for more characters per line
    \newenvironment{Shaded}{}{}
    \newcommand{\KeywordTok}[1]{\textcolor[rgb]{0.00,0.44,0.13}{\textbf{{#1}}}}
    \newcommand{\DataTypeTok}[1]{\textcolor[rgb]{0.56,0.13,0.00}{{#1}}}
    \newcommand{\DecValTok}[1]{\textcolor[rgb]{0.25,0.63,0.44}{{#1}}}
    \newcommand{\BaseNTok}[1]{\textcolor[rgb]{0.25,0.63,0.44}{{#1}}}
    \newcommand{\FloatTok}[1]{\textcolor[rgb]{0.25,0.63,0.44}{{#1}}}
    \newcommand{\CharTok}[1]{\textcolor[rgb]{0.25,0.44,0.63}{{#1}}}
    \newcommand{\StringTok}[1]{\textcolor[rgb]{0.25,0.44,0.63}{{#1}}}
    \newcommand{\CommentTok}[1]{\textcolor[rgb]{0.38,0.63,0.69}{\textit{{#1}}}}
    \newcommand{\OtherTok}[1]{\textcolor[rgb]{0.00,0.44,0.13}{{#1}}}
    \newcommand{\AlertTok}[1]{\textcolor[rgb]{1.00,0.00,0.00}{\textbf{{#1}}}}
    \newcommand{\FunctionTok}[1]{\textcolor[rgb]{0.02,0.16,0.49}{{#1}}}
    \newcommand{\RegionMarkerTok}[1]{{#1}}
    \newcommand{\ErrorTok}[1]{\textcolor[rgb]{1.00,0.00,0.00}{\textbf{{#1}}}}
    \newcommand{\NormalTok}[1]{{#1}}
    
    % Additional commands for more recent versions of Pandoc
    \newcommand{\ConstantTok}[1]{\textcolor[rgb]{0.53,0.00,0.00}{{#1}}}
    \newcommand{\SpecialCharTok}[1]{\textcolor[rgb]{0.25,0.44,0.63}{{#1}}}
    \newcommand{\VerbatimStringTok}[1]{\textcolor[rgb]{0.25,0.44,0.63}{{#1}}}
    \newcommand{\SpecialStringTok}[1]{\textcolor[rgb]{0.73,0.40,0.53}{{#1}}}
    \newcommand{\ImportTok}[1]{{#1}}
    \newcommand{\DocumentationTok}[1]{\textcolor[rgb]{0.73,0.13,0.13}{\textit{{#1}}}}
    \newcommand{\AnnotationTok}[1]{\textcolor[rgb]{0.38,0.63,0.69}{\textbf{\textit{{#1}}}}}
    \newcommand{\CommentVarTok}[1]{\textcolor[rgb]{0.38,0.63,0.69}{\textbf{\textit{{#1}}}}}
    \newcommand{\VariableTok}[1]{\textcolor[rgb]{0.10,0.09,0.49}{{#1}}}
    \newcommand{\ControlFlowTok}[1]{\textcolor[rgb]{0.00,0.44,0.13}{\textbf{{#1}}}}
    \newcommand{\OperatorTok}[1]{\textcolor[rgb]{0.40,0.40,0.40}{{#1}}}
    \newcommand{\BuiltInTok}[1]{{#1}}
    \newcommand{\ExtensionTok}[1]{{#1}}
    \newcommand{\PreprocessorTok}[1]{\textcolor[rgb]{0.74,0.48,0.00}{{#1}}}
    \newcommand{\AttributeTok}[1]{\textcolor[rgb]{0.49,0.56,0.16}{{#1}}}
    \newcommand{\InformationTok}[1]{\textcolor[rgb]{0.38,0.63,0.69}{\textbf{\textit{{#1}}}}}
    \newcommand{\WarningTok}[1]{\textcolor[rgb]{0.38,0.63,0.69}{\textbf{\textit{{#1}}}}}
    
    
    % Define a nice break command that doesn't care if a line doesn't already
    % exist.
    \def\br{\hspace*{\fill} \\* }
    % Math Jax compatability definitions
    \def\gt{>}
    \def\lt{<}
    % Document parameters
    \title{Exercise 2  Image Filters}
    
    
    

    % Pygments definitions
    
\makeatletter
\def\PY@reset{\let\PY@it=\relax \let\PY@bf=\relax%
    \let\PY@ul=\relax \let\PY@tc=\relax%
    \let\PY@bc=\relax \let\PY@ff=\relax}
\def\PY@tok#1{\csname PY@tok@#1\endcsname}
\def\PY@toks#1+{\ifx\relax#1\empty\else%
    \PY@tok{#1}\expandafter\PY@toks\fi}
\def\PY@do#1{\PY@bc{\PY@tc{\PY@ul{%
    \PY@it{\PY@bf{\PY@ff{#1}}}}}}}
\def\PY#1#2{\PY@reset\PY@toks#1+\relax+\PY@do{#2}}

\expandafter\def\csname PY@tok@gd\endcsname{\def\PY@tc##1{\textcolor[rgb]{0.63,0.00,0.00}{##1}}}
\expandafter\def\csname PY@tok@gu\endcsname{\let\PY@bf=\textbf\def\PY@tc##1{\textcolor[rgb]{0.50,0.00,0.50}{##1}}}
\expandafter\def\csname PY@tok@gt\endcsname{\def\PY@tc##1{\textcolor[rgb]{0.00,0.27,0.87}{##1}}}
\expandafter\def\csname PY@tok@gs\endcsname{\let\PY@bf=\textbf}
\expandafter\def\csname PY@tok@gr\endcsname{\def\PY@tc##1{\textcolor[rgb]{1.00,0.00,0.00}{##1}}}
\expandafter\def\csname PY@tok@cm\endcsname{\let\PY@it=\textit\def\PY@tc##1{\textcolor[rgb]{0.25,0.50,0.50}{##1}}}
\expandafter\def\csname PY@tok@vg\endcsname{\def\PY@tc##1{\textcolor[rgb]{0.10,0.09,0.49}{##1}}}
\expandafter\def\csname PY@tok@vi\endcsname{\def\PY@tc##1{\textcolor[rgb]{0.10,0.09,0.49}{##1}}}
\expandafter\def\csname PY@tok@vm\endcsname{\def\PY@tc##1{\textcolor[rgb]{0.10,0.09,0.49}{##1}}}
\expandafter\def\csname PY@tok@mh\endcsname{\def\PY@tc##1{\textcolor[rgb]{0.40,0.40,0.40}{##1}}}
\expandafter\def\csname PY@tok@cs\endcsname{\let\PY@it=\textit\def\PY@tc##1{\textcolor[rgb]{0.25,0.50,0.50}{##1}}}
\expandafter\def\csname PY@tok@ge\endcsname{\let\PY@it=\textit}
\expandafter\def\csname PY@tok@vc\endcsname{\def\PY@tc##1{\textcolor[rgb]{0.10,0.09,0.49}{##1}}}
\expandafter\def\csname PY@tok@il\endcsname{\def\PY@tc##1{\textcolor[rgb]{0.40,0.40,0.40}{##1}}}
\expandafter\def\csname PY@tok@go\endcsname{\def\PY@tc##1{\textcolor[rgb]{0.53,0.53,0.53}{##1}}}
\expandafter\def\csname PY@tok@cp\endcsname{\def\PY@tc##1{\textcolor[rgb]{0.74,0.48,0.00}{##1}}}
\expandafter\def\csname PY@tok@gi\endcsname{\def\PY@tc##1{\textcolor[rgb]{0.00,0.63,0.00}{##1}}}
\expandafter\def\csname PY@tok@gh\endcsname{\let\PY@bf=\textbf\def\PY@tc##1{\textcolor[rgb]{0.00,0.00,0.50}{##1}}}
\expandafter\def\csname PY@tok@ni\endcsname{\let\PY@bf=\textbf\def\PY@tc##1{\textcolor[rgb]{0.60,0.60,0.60}{##1}}}
\expandafter\def\csname PY@tok@nl\endcsname{\def\PY@tc##1{\textcolor[rgb]{0.63,0.63,0.00}{##1}}}
\expandafter\def\csname PY@tok@nn\endcsname{\let\PY@bf=\textbf\def\PY@tc##1{\textcolor[rgb]{0.00,0.00,1.00}{##1}}}
\expandafter\def\csname PY@tok@no\endcsname{\def\PY@tc##1{\textcolor[rgb]{0.53,0.00,0.00}{##1}}}
\expandafter\def\csname PY@tok@na\endcsname{\def\PY@tc##1{\textcolor[rgb]{0.49,0.56,0.16}{##1}}}
\expandafter\def\csname PY@tok@nb\endcsname{\def\PY@tc##1{\textcolor[rgb]{0.00,0.50,0.00}{##1}}}
\expandafter\def\csname PY@tok@nc\endcsname{\let\PY@bf=\textbf\def\PY@tc##1{\textcolor[rgb]{0.00,0.00,1.00}{##1}}}
\expandafter\def\csname PY@tok@nd\endcsname{\def\PY@tc##1{\textcolor[rgb]{0.67,0.13,1.00}{##1}}}
\expandafter\def\csname PY@tok@ne\endcsname{\let\PY@bf=\textbf\def\PY@tc##1{\textcolor[rgb]{0.82,0.25,0.23}{##1}}}
\expandafter\def\csname PY@tok@nf\endcsname{\def\PY@tc##1{\textcolor[rgb]{0.00,0.00,1.00}{##1}}}
\expandafter\def\csname PY@tok@si\endcsname{\let\PY@bf=\textbf\def\PY@tc##1{\textcolor[rgb]{0.73,0.40,0.53}{##1}}}
\expandafter\def\csname PY@tok@s2\endcsname{\def\PY@tc##1{\textcolor[rgb]{0.73,0.13,0.13}{##1}}}
\expandafter\def\csname PY@tok@nt\endcsname{\let\PY@bf=\textbf\def\PY@tc##1{\textcolor[rgb]{0.00,0.50,0.00}{##1}}}
\expandafter\def\csname PY@tok@nv\endcsname{\def\PY@tc##1{\textcolor[rgb]{0.10,0.09,0.49}{##1}}}
\expandafter\def\csname PY@tok@s1\endcsname{\def\PY@tc##1{\textcolor[rgb]{0.73,0.13,0.13}{##1}}}
\expandafter\def\csname PY@tok@dl\endcsname{\def\PY@tc##1{\textcolor[rgb]{0.73,0.13,0.13}{##1}}}
\expandafter\def\csname PY@tok@ch\endcsname{\let\PY@it=\textit\def\PY@tc##1{\textcolor[rgb]{0.25,0.50,0.50}{##1}}}
\expandafter\def\csname PY@tok@m\endcsname{\def\PY@tc##1{\textcolor[rgb]{0.40,0.40,0.40}{##1}}}
\expandafter\def\csname PY@tok@gp\endcsname{\let\PY@bf=\textbf\def\PY@tc##1{\textcolor[rgb]{0.00,0.00,0.50}{##1}}}
\expandafter\def\csname PY@tok@sh\endcsname{\def\PY@tc##1{\textcolor[rgb]{0.73,0.13,0.13}{##1}}}
\expandafter\def\csname PY@tok@ow\endcsname{\let\PY@bf=\textbf\def\PY@tc##1{\textcolor[rgb]{0.67,0.13,1.00}{##1}}}
\expandafter\def\csname PY@tok@sx\endcsname{\def\PY@tc##1{\textcolor[rgb]{0.00,0.50,0.00}{##1}}}
\expandafter\def\csname PY@tok@bp\endcsname{\def\PY@tc##1{\textcolor[rgb]{0.00,0.50,0.00}{##1}}}
\expandafter\def\csname PY@tok@c1\endcsname{\let\PY@it=\textit\def\PY@tc##1{\textcolor[rgb]{0.25,0.50,0.50}{##1}}}
\expandafter\def\csname PY@tok@fm\endcsname{\def\PY@tc##1{\textcolor[rgb]{0.00,0.00,1.00}{##1}}}
\expandafter\def\csname PY@tok@o\endcsname{\def\PY@tc##1{\textcolor[rgb]{0.40,0.40,0.40}{##1}}}
\expandafter\def\csname PY@tok@kc\endcsname{\let\PY@bf=\textbf\def\PY@tc##1{\textcolor[rgb]{0.00,0.50,0.00}{##1}}}
\expandafter\def\csname PY@tok@c\endcsname{\let\PY@it=\textit\def\PY@tc##1{\textcolor[rgb]{0.25,0.50,0.50}{##1}}}
\expandafter\def\csname PY@tok@mf\endcsname{\def\PY@tc##1{\textcolor[rgb]{0.40,0.40,0.40}{##1}}}
\expandafter\def\csname PY@tok@err\endcsname{\def\PY@bc##1{\setlength{\fboxsep}{0pt}\fcolorbox[rgb]{1.00,0.00,0.00}{1,1,1}{\strut ##1}}}
\expandafter\def\csname PY@tok@mb\endcsname{\def\PY@tc##1{\textcolor[rgb]{0.40,0.40,0.40}{##1}}}
\expandafter\def\csname PY@tok@ss\endcsname{\def\PY@tc##1{\textcolor[rgb]{0.10,0.09,0.49}{##1}}}
\expandafter\def\csname PY@tok@sr\endcsname{\def\PY@tc##1{\textcolor[rgb]{0.73,0.40,0.53}{##1}}}
\expandafter\def\csname PY@tok@mo\endcsname{\def\PY@tc##1{\textcolor[rgb]{0.40,0.40,0.40}{##1}}}
\expandafter\def\csname PY@tok@kd\endcsname{\let\PY@bf=\textbf\def\PY@tc##1{\textcolor[rgb]{0.00,0.50,0.00}{##1}}}
\expandafter\def\csname PY@tok@mi\endcsname{\def\PY@tc##1{\textcolor[rgb]{0.40,0.40,0.40}{##1}}}
\expandafter\def\csname PY@tok@kn\endcsname{\let\PY@bf=\textbf\def\PY@tc##1{\textcolor[rgb]{0.00,0.50,0.00}{##1}}}
\expandafter\def\csname PY@tok@cpf\endcsname{\let\PY@it=\textit\def\PY@tc##1{\textcolor[rgb]{0.25,0.50,0.50}{##1}}}
\expandafter\def\csname PY@tok@kr\endcsname{\let\PY@bf=\textbf\def\PY@tc##1{\textcolor[rgb]{0.00,0.50,0.00}{##1}}}
\expandafter\def\csname PY@tok@s\endcsname{\def\PY@tc##1{\textcolor[rgb]{0.73,0.13,0.13}{##1}}}
\expandafter\def\csname PY@tok@kp\endcsname{\def\PY@tc##1{\textcolor[rgb]{0.00,0.50,0.00}{##1}}}
\expandafter\def\csname PY@tok@w\endcsname{\def\PY@tc##1{\textcolor[rgb]{0.73,0.73,0.73}{##1}}}
\expandafter\def\csname PY@tok@kt\endcsname{\def\PY@tc##1{\textcolor[rgb]{0.69,0.00,0.25}{##1}}}
\expandafter\def\csname PY@tok@sc\endcsname{\def\PY@tc##1{\textcolor[rgb]{0.73,0.13,0.13}{##1}}}
\expandafter\def\csname PY@tok@sb\endcsname{\def\PY@tc##1{\textcolor[rgb]{0.73,0.13,0.13}{##1}}}
\expandafter\def\csname PY@tok@sa\endcsname{\def\PY@tc##1{\textcolor[rgb]{0.73,0.13,0.13}{##1}}}
\expandafter\def\csname PY@tok@k\endcsname{\let\PY@bf=\textbf\def\PY@tc##1{\textcolor[rgb]{0.00,0.50,0.00}{##1}}}
\expandafter\def\csname PY@tok@se\endcsname{\let\PY@bf=\textbf\def\PY@tc##1{\textcolor[rgb]{0.73,0.40,0.13}{##1}}}
\expandafter\def\csname PY@tok@sd\endcsname{\let\PY@it=\textit\def\PY@tc##1{\textcolor[rgb]{0.73,0.13,0.13}{##1}}}

\def\PYZbs{\char`\\}
\def\PYZus{\char`\_}
\def\PYZob{\char`\{}
\def\PYZcb{\char`\}}
\def\PYZca{\char`\^}
\def\PYZam{\char`\&}
\def\PYZlt{\char`\<}
\def\PYZgt{\char`\>}
\def\PYZsh{\char`\#}
\def\PYZpc{\char`\%}
\def\PYZdl{\char`\$}
\def\PYZhy{\char`\-}
\def\PYZsq{\char`\'}
\def\PYZdq{\char`\"}
\def\PYZti{\char`\~}
% for compatibility with earlier versions
\def\PYZat{@}
\def\PYZlb{[}
\def\PYZrb{]}
\makeatother


    % Exact colors from NB
    \definecolor{incolor}{rgb}{0.0, 0.0, 0.5}
    \definecolor{outcolor}{rgb}{0.545, 0.0, 0.0}



    
    % Prevent overflowing lines due to hard-to-break entities
    \sloppy 
    % Setup hyperref package
    \hypersetup{
      breaklinks=true,  % so long urls are correctly broken across lines
      colorlinks=true,
      urlcolor=urlcolor,
      linkcolor=linkcolor,
      citecolor=citecolor,
      }
    % Slightly bigger margins than the latex defaults
    
    \geometry{verbose,tmargin=1in,bmargin=1in,lmargin=1in,rmargin=1in}
    
    

    \begin{document}
    
    
    \maketitle
    
    

    
    ** Xiangyi Cheng (xxc283)**

    \hypertarget{concept-and-backgroud}{%
\section{Concept and Backgroud}\label{concept-and-backgroud}}

    Discrete cosine transform (DCT), as discussed in Exericse 1, transfers
image data from spatial space to frequential space. Different views of
image information are given by implementing the transform. Once having
the frequency basis functions, filtering is applied to the image by
modifying the coeffiency of each basis to achieve the processing goal,
such as image compression.

    \hypertarget{implement}{%
\section{Implement}\label{implement}}

    To develop the low-pass and high-pass filter, what we need to do is
simply zeroing different zones of the DCT coefficients.

First step is to add libraies which will be used inside the code.

    \begin{Verbatim}[commandchars=\\\{\}]
{\color{incolor}In [{\color{incolor}2}]:} \PY{k+kn}{import} \PY{n+nn}{numpy} \PY{k+kn}{as} \PY{n+nn}{np}
        \PY{k+kn}{import} \PY{n+nn}{matplotlib.pyplot} \PY{k+kn}{as} \PY{n+nn}{plt}
        \PY{k+kn}{from} \PY{n+nn}{scipy.fftpack} \PY{k+kn}{import} \PY{n}{dct}\PY{p}{,}\PY{n}{idct}
        \PY{k+kn}{import} \PY{n+nn}{cv2}
        \PY{k+kn}{import} \PY{n+nn}{math}
\end{Verbatim}


    Since only 1-D DCT and inverse cosine transform (iDCT) is provided by
Python, it would be better to come up with our own 2-D DCT and iDCT by
self-defined functions.

    \begin{Verbatim}[commandchars=\\\{\}]
{\color{incolor}In [{\color{incolor}3}]:} \PY{c+c1}{\PYZsh{} define the function to do dct and idct in 2D space}
        \PY{k}{def} \PY{n+nf}{dct2}\PY{p}{(}\PY{n}{image}\PY{p}{)}\PY{p}{:}
        	\PY{k}{return} \PY{n}{dct}\PY{p}{(}\PY{n}{dct}\PY{p}{(}\PY{n}{image}\PY{o}{.}\PY{n}{T}\PY{p}{,}\PY{n}{norm}\PY{o}{=}\PY{l+s+s1}{\PYZsq{}}\PY{l+s+s1}{ortho}\PY{l+s+s1}{\PYZsq{}}\PY{p}{)}\PY{o}{.}\PY{n}{T}\PY{p}{,}\PY{n}{norm}\PY{o}{=}\PY{l+s+s1}{\PYZsq{}}\PY{l+s+s1}{ortho}\PY{l+s+s1}{\PYZsq{}}\PY{p}{)}
        
        \PY{k}{def} \PY{n+nf}{idct2}\PY{p}{(}\PY{n}{dctmatrix}\PY{p}{)}\PY{p}{:}
        	\PY{k}{return} \PY{n}{idct}\PY{p}{(}\PY{n}{idct}\PY{p}{(}\PY{n}{dctmatrix}\PY{o}{.}\PY{n}{T}\PY{p}{,}\PY{n}{norm}\PY{o}{=}\PY{l+s+s1}{\PYZsq{}}\PY{l+s+s1}{ortho}\PY{l+s+s1}{\PYZsq{}}\PY{p}{)}\PY{o}{.}\PY{n}{T}\PY{p}{,}\PY{n}{norm}\PY{o}{=}\PY{l+s+s1}{\PYZsq{}}\PY{l+s+s1}{ortho}\PY{l+s+s1}{\PYZsq{}}\PY{p}{)}
\end{Verbatim}


    OpenCV is used to load and write the image. Then the defined function is
called to get the 2-D DCT of the image so that the image data in
frequential space is obtained.

    \begin{Verbatim}[commandchars=\\\{\}]
{\color{incolor}In [{\color{incolor}28}]:} \PY{c+c1}{\PYZsh{}img=cv2.imread(\PYZsq{}sydney.jpg\PYZsq{},cv2.IMREAD\PYZus{}GRAYSCALE)}
         \PY{n}{img}\PY{o}{=}\PY{n}{cv2}\PY{o}{.}\PY{n}{imread}\PY{p}{(}\PY{l+s+s1}{\PYZsq{}}\PY{l+s+s1}{sydney.jpg}\PY{l+s+s1}{\PYZsq{}}\PY{p}{,}\PY{l+m+mi}{0}\PY{p}{)}
         \PY{n}{gray\PYZus{}img}\PY{o}{=}\PY{n}{cv2}\PY{o}{.}\PY{n}{imwrite}\PY{p}{(}\PY{l+s+s1}{\PYZsq{}}\PY{l+s+s1}{gray\PYZus{}sydney.jpg}\PY{l+s+s1}{\PYZsq{}}\PY{p}{,}\PY{n}{img}\PY{p}{)}
         
         \PY{c+c1}{\PYZsh{} do the dct to transfer the image from spatial space into frequency space}
         \PY{n}{dct\PYZus{}img}\PY{o}{=}\PY{n}{dct2}\PY{p}{(}\PY{n}{img}\PY{p}{)}
         \PY{n}{m}\PY{p}{,}\PY{n}{n}\PY{o}{=}\PY{n}{dct\PYZus{}img}\PY{o}{.}\PY{n}{shape}
\end{Verbatim}


    Generating the low-pass filter is achieved by removing the high
frequency zone. In this track, simply zeroing the coeffient in the high
frequency is totally enough. As talked in exercise 1, the frequency is
increasing by comparing each row and each column. Low frequency
relatively exists in the top left corner of the matrix while high
frequency appears in the bottom right corner.

Therefore, zeroing the row 10 to the end and column 10 to end of the
coefficent matrix is to apply the low pass filter.

    \begin{Verbatim}[commandchars=\\\{\}]
{\color{incolor}In [{\color{incolor}29}]:} \PY{n}{low\PYZus{}filter}\PY{o}{=}\PY{n}{dct\PYZus{}img}
         \PY{n}{low\PYZus{}filter}\PY{p}{[}\PY{l+m+mi}{10}\PY{p}{:}\PY{p}{,}\PY{l+m+mi}{10}\PY{p}{:}\PY{p}{]}\PY{o}{=}\PY{l+m+mi}{0}
\end{Verbatim}


    Inversing the processed matrix back to x,y space is the next step. Then
we can compare the image with low-pass filter with the original image to
see the differences between them.

    \begin{Verbatim}[commandchars=\\\{\}]
{\color{incolor}In [{\color{incolor}30}]:} \PY{n}{low\PYZus{}idct}\PY{o}{=}\PY{n}{idct2}\PY{p}{(}\PY{n}{low\PYZus{}filter}\PY{p}{)}
         \PY{n}{diff\PYZus{}low}\PY{o}{=}\PY{n+nb}{abs}\PY{p}{(}\PY{n}{low\PYZus{}idct}\PY{o}{\PYZhy{}}\PY{n}{img}\PY{p}{)}
         \PY{n}{cv2}\PY{o}{.}\PY{n}{imwrite}\PY{p}{(}\PY{l+s+s1}{\PYZsq{}}\PY{l+s+s1}{low\PYZus{}filtered\PYZus{}sydney.jpg}\PY{l+s+s1}{\PYZsq{}}\PY{p}{,}\PY{n}{low\PYZus{}idct}\PY{p}{)}
         \PY{n}{cv2}\PY{o}{.}\PY{n}{imwrite}\PY{p}{(}\PY{l+s+s1}{\PYZsq{}}\PY{l+s+s1}{diff\PYZus{}low.jpg}\PY{l+s+s1}{\PYZsq{}}\PY{p}{,}\PY{n}{diff\PYZus{}low}\PY{p}{)}
\end{Verbatim}


\begin{Verbatim}[commandchars=\\\{\}]
{\color{outcolor}Out[{\color{outcolor}30}]:} True
\end{Verbatim}
            
    Follow the basis idea of applying low-pass filter to obtain the
high-pass filter. This time the coeffient of the lower frequency zone
should be zero. Therefore, zero the row 0 to 100 and column 0 to 100.

    \begin{Verbatim}[commandchars=\\\{\}]
{\color{incolor}In [{\color{incolor}31}]:} \PY{n}{high\PYZus{}filter}\PY{o}{=}\PY{n}{dct\PYZus{}img}
         \PY{n}{high\PYZus{}filter}\PY{p}{[}\PY{l+m+mi}{0}\PY{p}{:}\PY{l+m+mi}{100}\PY{p}{,}\PY{l+m+mi}{0}\PY{p}{:}\PY{l+m+mi}{100}\PY{p}{]}\PY{o}{=}\PY{l+m+mi}{0}
         \PY{n}{high\PYZus{}idct}\PY{o}{=}\PY{n}{idct2}\PY{p}{(}\PY{n}{high\PYZus{}filter}\PY{p}{)}
         \PY{n}{diff\PYZus{}high}\PY{o}{=}\PY{n+nb}{abs}\PY{p}{(}\PY{n}{high\PYZus{}idct}\PY{o}{\PYZhy{}}\PY{n}{img}\PY{p}{)}
         \PY{n}{cv2}\PY{o}{.}\PY{n}{imwrite}\PY{p}{(}\PY{l+s+s1}{\PYZsq{}}\PY{l+s+s1}{high\PYZus{}filtered\PYZus{}sydney.jpg}\PY{l+s+s1}{\PYZsq{}}\PY{p}{,}\PY{n}{high\PYZus{}idct}\PY{p}{)}
         \PY{n}{cv2}\PY{o}{.}\PY{n}{imwrite}\PY{p}{(}\PY{l+s+s1}{\PYZsq{}}\PY{l+s+s1}{diff\PYZus{}high.jpg}\PY{l+s+s1}{\PYZsq{}}\PY{p}{,}\PY{n}{diff\PYZus{}high}\PY{p}{)}
\end{Verbatim}


\begin{Verbatim}[commandchars=\\\{\}]
{\color{outcolor}Out[{\color{outcolor}31}]:} True
\end{Verbatim}
            
    To illustrate the results and differences between images, the processed
images and differences are plotted out.

    \begin{Verbatim}[commandchars=\\\{\}]
{\color{incolor}In [{\color{incolor}32}]:} \PY{c+c1}{\PYZsh{} show the original image}
         \PY{n}{plt}\PY{o}{.}\PY{n}{figure}\PY{p}{(}\PY{n}{figsize}\PY{o}{=}\PY{p}{(}\PY{l+m+mi}{20}\PY{p}{,}\PY{l+m+mi}{20}\PY{p}{)}\PY{p}{)}
         \PY{n}{original\PYZus{}img}\PY{o}{=}\PY{n}{cv2}\PY{o}{.}\PY{n}{imread}\PY{p}{(}\PY{l+s+s1}{\PYZsq{}}\PY{l+s+s1}{gray\PYZus{}sydney.jpg}\PY{l+s+s1}{\PYZsq{}}\PY{p}{)}
         \PY{n}{original\PYZus{}img}\PY{o}{=}\PY{n}{cv2}\PY{o}{.}\PY{n}{cvtColor}\PY{p}{(}\PY{n}{original\PYZus{}img}\PY{p}{,}\PY{n}{cv2}\PY{o}{.}\PY{n}{COLOR\PYZus{}BGR2RGB}\PY{p}{)}
         \PY{n}{plt}\PY{o}{.}\PY{n}{subplot}\PY{p}{(}\PY{l+m+mi}{2}\PY{p}{,}\PY{l+m+mi}{3}\PY{p}{,}\PY{l+m+mi}{1}\PY{p}{)}
         \PY{n}{plt}\PY{o}{.}\PY{n}{imshow}\PY{p}{(}\PY{n}{original\PYZus{}img}\PY{p}{)}
         \PY{n}{plt}\PY{o}{.}\PY{n}{title}\PY{p}{(}\PY{l+s+s1}{\PYZsq{}}\PY{l+s+s1}{original image}\PY{l+s+s1}{\PYZsq{}}\PY{p}{)}
         
         
         \PY{c+c1}{\PYZsh{} show the processed image by the low pass filter}
         \PY{n}{low\PYZus{}filtered\PYZus{}sydney}\PY{o}{=}\PY{n}{cv2}\PY{o}{.}\PY{n}{imread}\PY{p}{(}\PY{l+s+s1}{\PYZsq{}}\PY{l+s+s1}{low\PYZus{}filtered\PYZus{}sydney.jpg}\PY{l+s+s1}{\PYZsq{}}\PY{p}{)}
         \PY{n}{low\PYZus{}filtered\PYZus{}sydney}\PY{o}{=}\PY{n}{cv2}\PY{o}{.}\PY{n}{cvtColor}\PY{p}{(}\PY{n}{low\PYZus{}filtered\PYZus{}sydney}\PY{p}{,}\PY{n}{cv2}\PY{o}{.}\PY{n}{COLOR\PYZus{}BGR2RGB}\PY{p}{)}
         \PY{n}{plt}\PY{o}{.}\PY{n}{subplot}\PY{p}{(}\PY{l+m+mi}{2}\PY{p}{,}\PY{l+m+mi}{3}\PY{p}{,}\PY{l+m+mi}{2}\PY{p}{)}
         \PY{n}{plt}\PY{o}{.}\PY{n}{imshow}\PY{p}{(}\PY{n}{low\PYZus{}filtered\PYZus{}sydney}\PY{p}{)}
         \PY{n}{plt}\PY{o}{.}\PY{n}{title}\PY{p}{(}\PY{l+s+s1}{\PYZsq{}}\PY{l+s+s1}{with low fillter}\PY{l+s+s1}{\PYZsq{}}\PY{p}{)}
         
         \PY{c+c1}{\PYZsh{} show the difference between the orignal image and processed image applied low pass filter}
         \PY{n}{diff\PYZus{}low}\PY{o}{=}\PY{n}{cv2}\PY{o}{.}\PY{n}{imread}\PY{p}{(}\PY{l+s+s1}{\PYZsq{}}\PY{l+s+s1}{diff\PYZus{}low.jpg}\PY{l+s+s1}{\PYZsq{}}\PY{p}{)}
         \PY{n}{diff\PYZus{}low}\PY{o}{=}\PY{n}{cv2}\PY{o}{.}\PY{n}{cvtColor}\PY{p}{(}\PY{n}{diff\PYZus{}low}\PY{p}{,}\PY{n}{cv2}\PY{o}{.}\PY{n}{COLOR\PYZus{}BGR2RGB}\PY{p}{)}
         \PY{n}{plt}\PY{o}{.}\PY{n}{subplot}\PY{p}{(}\PY{l+m+mi}{2}\PY{p}{,}\PY{l+m+mi}{3}\PY{p}{,}\PY{l+m+mi}{5}\PY{p}{)}
         \PY{n}{plt}\PY{o}{.}\PY{n}{imshow}\PY{p}{(}\PY{n}{diff\PYZus{}low}\PY{p}{)}
         \PY{n}{plt}\PY{o}{.}\PY{n}{title}\PY{p}{(}\PY{l+s+s1}{\PYZsq{}}\PY{l+s+s1}{difference}\PY{l+s+s1}{\PYZsq{}}\PY{p}{)}
         
         
         \PY{c+c1}{\PYZsh{} show the processed image by the high pass filter}
         \PY{n}{high\PYZus{}filtered\PYZus{}sydney}\PY{o}{=}\PY{n}{cv2}\PY{o}{.}\PY{n}{imread}\PY{p}{(}\PY{l+s+s1}{\PYZsq{}}\PY{l+s+s1}{high\PYZus{}filtered\PYZus{}sydney.jpg}\PY{l+s+s1}{\PYZsq{}}\PY{p}{)}
         \PY{n}{high\PYZus{}filtered\PYZus{}sydney}\PY{o}{=}\PY{n}{cv2}\PY{o}{.}\PY{n}{cvtColor}\PY{p}{(}\PY{n}{high\PYZus{}filtered\PYZus{}sydney}\PY{p}{,}\PY{n}{cv2}\PY{o}{.}\PY{n}{COLOR\PYZus{}BGR2RGB}\PY{p}{)}
         \PY{n}{plt}\PY{o}{.}\PY{n}{subplot}\PY{p}{(}\PY{l+m+mi}{2}\PY{p}{,}\PY{l+m+mi}{3}\PY{p}{,}\PY{l+m+mi}{3}\PY{p}{)}
         \PY{n}{plt}\PY{o}{.}\PY{n}{imshow}\PY{p}{(}\PY{n}{high\PYZus{}filtered\PYZus{}sydney}\PY{p}{)}
         \PY{n}{plt}\PY{o}{.}\PY{n}{title}\PY{p}{(}\PY{l+s+s1}{\PYZsq{}}\PY{l+s+s1}{with high fillter}\PY{l+s+s1}{\PYZsq{}}\PY{p}{)}
         
         \PY{c+c1}{\PYZsh{} show the difference between the orignal image and processed image applied high pass filter}
         \PY{n}{diff\PYZus{}high}\PY{o}{=}\PY{n}{cv2}\PY{o}{.}\PY{n}{imread}\PY{p}{(}\PY{l+s+s1}{\PYZsq{}}\PY{l+s+s1}{diff\PYZus{}high.jpg}\PY{l+s+s1}{\PYZsq{}}\PY{p}{)}
         \PY{n}{diff\PYZus{}high}\PY{o}{=}\PY{n}{cv2}\PY{o}{.}\PY{n}{cvtColor}\PY{p}{(}\PY{n}{diff\PYZus{}high}\PY{p}{,}\PY{n}{cv2}\PY{o}{.}\PY{n}{COLOR\PYZus{}BGR2RGB}\PY{p}{)}
         \PY{n}{plt}\PY{o}{.}\PY{n}{subplot}\PY{p}{(}\PY{l+m+mi}{2}\PY{p}{,}\PY{l+m+mi}{3}\PY{p}{,}\PY{l+m+mi}{6}\PY{p}{)}
         \PY{n}{plt}\PY{o}{.}\PY{n}{imshow}\PY{p}{(}\PY{n}{diff\PYZus{}high}\PY{p}{)}
         \PY{n}{plt}\PY{o}{.}\PY{n}{title}\PY{p}{(}\PY{l+s+s1}{\PYZsq{}}\PY{l+s+s1}{difference}\PY{l+s+s1}{\PYZsq{}}\PY{p}{)}
         \PY{n}{plt}\PY{o}{.}\PY{n}{show}\PY{p}{(}\PY{p}{)}
\end{Verbatim}


    \begin{center}
    \adjustimage{max size={0.9\linewidth}{0.9\paperheight}}{output_17_0.png}
    \end{center}
    { \hspace*{\fill} \\}
    
    To have a clear sense that how the filters work, the size of zero zones
are modified to 40 for the low-pass filter and 40 for the high-pass
filter.

    \begin{Verbatim}[commandchars=\\\{\}]
{\color{incolor}In [{\color{incolor}25}]:} \PY{c+c1}{\PYZsh{} apply the low pass filter}
         \PY{n}{low\PYZus{}filter}\PY{o}{=}\PY{n}{dct\PYZus{}img}
         \PY{n}{low\PYZus{}filter}\PY{p}{[}\PY{l+m+mi}{40}\PY{p}{:}\PY{p}{,}\PY{l+m+mi}{40}\PY{p}{:}\PY{p}{]}\PY{o}{=}\PY{l+m+mi}{0}
         \PY{n}{low\PYZus{}idct}\PY{o}{=}\PY{n}{idct2}\PY{p}{(}\PY{n}{low\PYZus{}filter}\PY{p}{)}
         \PY{n}{diff\PYZus{}low}\PY{o}{=}\PY{n+nb}{abs}\PY{p}{(}\PY{n}{low\PYZus{}idct}\PY{o}{\PYZhy{}}\PY{n}{img}\PY{p}{)}
         \PY{n}{cv2}\PY{o}{.}\PY{n}{imwrite}\PY{p}{(}\PY{l+s+s1}{\PYZsq{}}\PY{l+s+s1}{low\PYZus{}filtered\PYZus{}sydney.jpg}\PY{l+s+s1}{\PYZsq{}}\PY{p}{,}\PY{n}{low\PYZus{}idct}\PY{p}{)}
         \PY{n}{cv2}\PY{o}{.}\PY{n}{imwrite}\PY{p}{(}\PY{l+s+s1}{\PYZsq{}}\PY{l+s+s1}{diff\PYZus{}low.jpg}\PY{l+s+s1}{\PYZsq{}}\PY{p}{,}\PY{n}{diff\PYZus{}low}\PY{p}{)}
         
         \PY{n}{dct\PYZus{}img}\PY{o}{=}\PY{n}{dct2}\PY{p}{(}\PY{n}{img}\PY{p}{)}
         \PY{n}{m}\PY{p}{,}\PY{n}{n}\PY{o}{=}\PY{n}{dct\PYZus{}img}\PY{o}{.}\PY{n}{shape}
         \PY{c+c1}{\PYZsh{} apply the high pass filter}
         \PY{n}{high\PYZus{}filter}\PY{o}{=}\PY{n}{dct\PYZus{}img}
         \PY{n}{high\PYZus{}filter}\PY{p}{[}\PY{l+m+mi}{0}\PY{p}{:}\PY{l+m+mi}{40}\PY{p}{,}\PY{l+m+mi}{0}\PY{p}{:}\PY{l+m+mi}{40}\PY{p}{]}\PY{o}{=}\PY{l+m+mi}{0}
         \PY{n}{high\PYZus{}idct}\PY{o}{=}\PY{n}{idct2}\PY{p}{(}\PY{n}{high\PYZus{}filter}\PY{p}{)}
         \PY{n}{diff\PYZus{}high}\PY{o}{=}\PY{n+nb}{abs}\PY{p}{(}\PY{n}{high\PYZus{}idct}\PY{o}{\PYZhy{}}\PY{n}{img}\PY{p}{)}
         \PY{n}{cv2}\PY{o}{.}\PY{n}{imwrite}\PY{p}{(}\PY{l+s+s1}{\PYZsq{}}\PY{l+s+s1}{high\PYZus{}filtered\PYZus{}sydney.jpg}\PY{l+s+s1}{\PYZsq{}}\PY{p}{,}\PY{n}{high\PYZus{}idct}\PY{p}{)}
         \PY{n}{cv2}\PY{o}{.}\PY{n}{imwrite}\PY{p}{(}\PY{l+s+s1}{\PYZsq{}}\PY{l+s+s1}{diff\PYZus{}high.jpg}\PY{l+s+s1}{\PYZsq{}}\PY{p}{,}\PY{n}{diff\PYZus{}high}\PY{p}{)}
\end{Verbatim}


\begin{Verbatim}[commandchars=\\\{\}]
{\color{outcolor}Out[{\color{outcolor}25}]:} True
\end{Verbatim}
            
    Plot the results and show the results.

    \begin{Verbatim}[commandchars=\\\{\}]
{\color{incolor}In [{\color{incolor}30}]:} \PY{c+c1}{\PYZsh{} show the original image}
         \PY{n}{plt}\PY{o}{.}\PY{n}{figure}\PY{p}{(}\PY{n}{figsize}\PY{o}{=}\PY{p}{(}\PY{l+m+mi}{20}\PY{p}{,}\PY{l+m+mi}{20}\PY{p}{)}\PY{p}{)}
         \PY{n}{original\PYZus{}img}\PY{o}{=}\PY{n}{cv2}\PY{o}{.}\PY{n}{imread}\PY{p}{(}\PY{l+s+s1}{\PYZsq{}}\PY{l+s+s1}{gray\PYZus{}sydney.jpg}\PY{l+s+s1}{\PYZsq{}}\PY{p}{)}
         \PY{n}{original\PYZus{}img}\PY{o}{=}\PY{n}{cv2}\PY{o}{.}\PY{n}{cvtColor}\PY{p}{(}\PY{n}{original\PYZus{}img}\PY{p}{,}\PY{n}{cv2}\PY{o}{.}\PY{n}{COLOR\PYZus{}BGR2RGB}\PY{p}{)}
         \PY{n}{plt}\PY{o}{.}\PY{n}{subplot}\PY{p}{(}\PY{l+m+mi}{2}\PY{p}{,}\PY{l+m+mi}{3}\PY{p}{,}\PY{l+m+mi}{1}\PY{p}{)}
         \PY{n}{plt}\PY{o}{.}\PY{n}{imshow}\PY{p}{(}\PY{n}{original\PYZus{}img}\PY{p}{)}
         \PY{n}{plt}\PY{o}{.}\PY{n}{title}\PY{p}{(}\PY{l+s+s1}{\PYZsq{}}\PY{l+s+s1}{original image}\PY{l+s+s1}{\PYZsq{}}\PY{p}{)}
         
         
         \PY{c+c1}{\PYZsh{} show the processed image by the low pass filter}
         \PY{n}{low\PYZus{}filtered\PYZus{}sydney}\PY{o}{=}\PY{n}{cv2}\PY{o}{.}\PY{n}{imread}\PY{p}{(}\PY{l+s+s1}{\PYZsq{}}\PY{l+s+s1}{low\PYZus{}filtered\PYZus{}sydney.jpg}\PY{l+s+s1}{\PYZsq{}}\PY{p}{)}
         \PY{n}{low\PYZus{}filtered\PYZus{}sydney}\PY{o}{=}\PY{n}{cv2}\PY{o}{.}\PY{n}{cvtColor}\PY{p}{(}\PY{n}{low\PYZus{}filtered\PYZus{}sydney}\PY{p}{,}\PY{n}{cv2}\PY{o}{.}\PY{n}{COLOR\PYZus{}BGR2RGB}\PY{p}{)}
         \PY{n}{plt}\PY{o}{.}\PY{n}{subplot}\PY{p}{(}\PY{l+m+mi}{2}\PY{p}{,}\PY{l+m+mi}{3}\PY{p}{,}\PY{l+m+mi}{2}\PY{p}{)}
         \PY{n}{plt}\PY{o}{.}\PY{n}{imshow}\PY{p}{(}\PY{n}{low\PYZus{}filtered\PYZus{}sydney}\PY{p}{)}
         \PY{n}{plt}\PY{o}{.}\PY{n}{title}\PY{p}{(}\PY{l+s+s1}{\PYZsq{}}\PY{l+s+s1}{with low fillter}\PY{l+s+s1}{\PYZsq{}}\PY{p}{)}
         
         \PY{c+c1}{\PYZsh{} show the difference between the orignal image and processed image applied low pass filter}
         \PY{n}{diff\PYZus{}low}\PY{o}{=}\PY{n}{cv2}\PY{o}{.}\PY{n}{imread}\PY{p}{(}\PY{l+s+s1}{\PYZsq{}}\PY{l+s+s1}{diff\PYZus{}low.jpg}\PY{l+s+s1}{\PYZsq{}}\PY{p}{)}
         \PY{n}{dct\PYZus{}img}\PY{o}{=}\PY{n}{dct2}\PY{p}{(}\PY{n}{img}\PY{p}{)}
         \PY{n}{m}\PY{p}{,}\PY{n}{n}\PY{o}{=}\PY{n}{dct\PYZus{}img}\PY{o}{.}\PY{n}{shape}
         \PY{n}{diff\PYZus{}low}\PY{o}{=}\PY{n}{cv2}\PY{o}{.}\PY{n}{cvtColor}\PY{p}{(}\PY{n}{diff\PYZus{}low}\PY{p}{,}\PY{n}{cv2}\PY{o}{.}\PY{n}{COLOR\PYZus{}BGR2RGB}\PY{p}{)}
         \PY{n}{plt}\PY{o}{.}\PY{n}{subplot}\PY{p}{(}\PY{l+m+mi}{2}\PY{p}{,}\PY{l+m+mi}{3}\PY{p}{,}\PY{l+m+mi}{5}\PY{p}{)}
         \PY{n}{plt}\PY{o}{.}\PY{n}{imshow}\PY{p}{(}\PY{n}{diff\PYZus{}low}\PY{p}{)}
         \PY{n}{plt}\PY{o}{.}\PY{n}{title}\PY{p}{(}\PY{l+s+s1}{\PYZsq{}}\PY{l+s+s1}{difference}\PY{l+s+s1}{\PYZsq{}}\PY{p}{)}
         
         
         \PY{c+c1}{\PYZsh{} show the processed image by the high pass filter}
         \PY{n}{high\PYZus{}filtered\PYZus{}sydney}\PY{o}{=}\PY{n}{cv2}\PY{o}{.}\PY{n}{imread}\PY{p}{(}\PY{l+s+s1}{\PYZsq{}}\PY{l+s+s1}{high\PYZus{}filtered\PYZus{}sydney.jpg}\PY{l+s+s1}{\PYZsq{}}\PY{p}{)}
         \PY{n}{high\PYZus{}filtered\PYZus{}sydney}\PY{o}{=}\PY{n}{cv2}\PY{o}{.}\PY{n}{cvtColor}\PY{p}{(}\PY{n}{high\PYZus{}filtered\PYZus{}sydney}\PY{p}{,}\PY{n}{cv2}\PY{o}{.}\PY{n}{COLOR\PYZus{}BGR2RGB}\PY{p}{)}
         \PY{n}{plt}\PY{o}{.}\PY{n}{subplot}\PY{p}{(}\PY{l+m+mi}{2}\PY{p}{,}\PY{l+m+mi}{3}\PY{p}{,}\PY{l+m+mi}{3}\PY{p}{)}
         \PY{n}{plt}\PY{o}{.}\PY{n}{imshow}\PY{p}{(}\PY{n}{high\PYZus{}filtered\PYZus{}sydney}\PY{p}{)}
         \PY{n}{plt}\PY{o}{.}\PY{n}{title}\PY{p}{(}\PY{l+s+s1}{\PYZsq{}}\PY{l+s+s1}{with high fillter}\PY{l+s+s1}{\PYZsq{}}\PY{p}{)}
         
         \PY{c+c1}{\PYZsh{} show the difference between the orignal image and processed image applied high pass filter}
         \PY{n}{diff\PYZus{}high}\PY{o}{=}\PY{n}{cv2}\PY{o}{.}\PY{n}{imread}\PY{p}{(}\PY{l+s+s1}{\PYZsq{}}\PY{l+s+s1}{diff\PYZus{}high.jpg}\PY{l+s+s1}{\PYZsq{}}\PY{p}{)}
         \PY{n}{diff\PYZus{}high}\PY{o}{=}\PY{n}{cv2}\PY{o}{.}\PY{n}{cvtColor}\PY{p}{(}\PY{n}{diff\PYZus{}high}\PY{p}{,}\PY{n}{cv2}\PY{o}{.}\PY{n}{COLOR\PYZus{}BGR2RGB}\PY{p}{)}
         \PY{n}{plt}\PY{o}{.}\PY{n}{subplot}\PY{p}{(}\PY{l+m+mi}{2}\PY{p}{,}\PY{l+m+mi}{3}\PY{p}{,}\PY{l+m+mi}{6}\PY{p}{)}
         \PY{n}{plt}\PY{o}{.}\PY{n}{imshow}\PY{p}{(}\PY{n}{diff\PYZus{}high}\PY{p}{)}
         \PY{n}{plt}\PY{o}{.}\PY{n}{title}\PY{p}{(}\PY{l+s+s1}{\PYZsq{}}\PY{l+s+s1}{difference}\PY{l+s+s1}{\PYZsq{}}\PY{p}{)}\PY{l+m+mi}{10}
         \PY{n}{plt}\PY{o}{.}\PY{n}{show}\PY{p}{(}\PY{p}{)}
\end{Verbatim}


    \begin{center}
    \adjustimage{max size={0.9\linewidth}{0.9\paperheight}}{output_21_0.png}
    \end{center}
    { \hspace*{\fill} \\}
    
    \hypertarget{extension}{%
\section{Extension}\label{extension}}

    To get a band filter, only the information in the middle zone is kept.

    \begin{Verbatim}[commandchars=\\\{\}]
{\color{incolor}In [{\color{incolor}39}]:} \PY{c+c1}{\PYZsh{} apply the band pass filter}
         \PY{n}{band\PYZus{}filter}\PY{o}{=}\PY{n}{dct\PYZus{}img}
         \PY{n}{band\PYZus{}filter}\PY{p}{[}\PY{l+m+mi}{0}\PY{p}{:}\PY{l+m+mi}{20}\PY{p}{,}\PY{l+m+mi}{0}\PY{p}{:}\PY{l+m+mi}{20}\PY{p}{]}\PY{o}{=}\PY{l+m+mi}{0}
         \PY{n}{band\PYZus{}filter}\PY{p}{[}\PY{l+m+mi}{80}\PY{p}{:}\PY{p}{,}\PY{l+m+mi}{80}\PY{p}{:}\PY{p}{]}\PY{o}{=}\PY{l+m+mi}{0}
         \PY{n}{band\PYZus{}idct}\PY{o}{=}\PY{n}{idct2}\PY{p}{(}\PY{n}{band\PYZus{}filter}\PY{p}{)}
         \PY{n}{diff\PYZus{}band}\PY{o}{=}\PY{n+nb}{abs}\PY{p}{(}\PY{n}{band\PYZus{}idct}\PY{o}{\PYZhy{}}\PY{n}{img}\PY{p}{)}
         \PY{n}{cv2}\PY{o}{.}\PY{n}{imwrite}\PY{p}{(}\PY{l+s+s1}{\PYZsq{}}\PY{l+s+s1}{band\PYZus{}filtered\PYZus{}sydney.jpg}\PY{l+s+s1}{\PYZsq{}}\PY{p}{,}\PY{n}{band\PYZus{}idct}\PY{p}{)}
         \PY{n}{cv2}\PY{o}{.}\PY{n}{imwrite}\PY{p}{(}\PY{l+s+s1}{\PYZsq{}}\PY{l+s+s1}{diff\PYZus{}band.jpg}\PY{l+s+s1}{\PYZsq{}}\PY{p}{,}\PY{n}{diff\PYZus{}band}\PY{p}{)}
\end{Verbatim}


\begin{Verbatim}[commandchars=\\\{\}]
{\color{outcolor}Out[{\color{outcolor}39}]:} True
\end{Verbatim}
            
    Show the result.

    \begin{Verbatim}[commandchars=\\\{\}]
{\color{incolor}In [{\color{incolor}40}]:} \PY{c+c1}{\PYZsh{} show the original image}
         \PY{n}{plt}\PY{o}{.}\PY{n}{figure}\PY{p}{(}\PY{n}{figsize}\PY{o}{=}\PY{p}{(}\PY{l+m+mi}{20}\PY{p}{,}\PY{l+m+mi}{20}\PY{p}{)}\PY{p}{)}
         \PY{n}{original\PYZus{}img}\PY{o}{=}\PY{n}{cv2}\PY{o}{.}\PY{n}{imread}\PY{p}{(}\PY{l+s+s1}{\PYZsq{}}\PY{l+s+s1}{gray\PYZus{}sydney.jpg}\PY{l+s+s1}{\PYZsq{}}\PY{p}{)}
         \PY{n}{original\PYZus{}img}\PY{o}{=}\PY{n}{cv2}\PY{o}{.}\PY{n}{cvtColor}\PY{p}{(}\PY{n}{original\PYZus{}img}\PY{p}{,}\PY{n}{cv2}\PY{o}{.}\PY{n}{COLOR\PYZus{}BGR2RGB}\PY{p}{)}
         \PY{n}{plt}\PY{o}{.}\PY{n}{subplot}\PY{p}{(}\PY{l+m+mi}{1}\PY{p}{,}\PY{l+m+mi}{3}\PY{p}{,}\PY{l+m+mi}{1}\PY{p}{)}
         \PY{n}{plt}\PY{o}{.}\PY{n}{imshow}\PY{p}{(}\PY{n}{original\PYZus{}img}\PY{p}{)}
         \PY{n}{plt}\PY{o}{.}\PY{n}{title}\PY{p}{(}\PY{l+s+s1}{\PYZsq{}}\PY{l+s+s1}{original image}\PY{l+s+s1}{\PYZsq{}}\PY{p}{)}
         
         
         \PY{c+c1}{\PYZsh{} show the processed image by the band pass filter}
         \PY{n}{band\PYZus{}filtered\PYZus{}sydney}\PY{o}{=}\PY{n}{cv2}\PY{o}{.}\PY{n}{imread}\PY{p}{(}\PY{l+s+s1}{\PYZsq{}}\PY{l+s+s1}{band\PYZus{}filtered\PYZus{}sydney.jpg}\PY{l+s+s1}{\PYZsq{}}\PY{p}{)}
         \PY{n}{band\PYZus{}filtered\PYZus{}sydney}\PY{o}{=}\PY{n}{cv2}\PY{o}{.}\PY{n}{cvtColor}\PY{p}{(}\PY{n}{band\PYZus{}filtered\PYZus{}sydney}\PY{p}{,}\PY{n}{cv2}\PY{o}{.}\PY{n}{COLOR\PYZus{}BGR2RGB}\PY{p}{)}
         \PY{n}{plt}\PY{o}{.}\PY{n}{subplot}\PY{p}{(}\PY{l+m+mi}{1}\PY{p}{,}\PY{l+m+mi}{3}\PY{p}{,}\PY{l+m+mi}{2}\PY{p}{)}
         \PY{n}{plt}\PY{o}{.}\PY{n}{imshow}\PY{p}{(}\PY{n}{band\PYZus{}filtered\PYZus{}sydney}\PY{p}{)}
         \PY{n}{plt}\PY{o}{.}\PY{n}{title}\PY{p}{(}\PY{l+s+s1}{\PYZsq{}}\PY{l+s+s1}{with band fillter}\PY{l+s+s1}{\PYZsq{}}\PY{p}{)}
         
         \PY{c+c1}{\PYZsh{} show the difference between the orignal image and processed image applied band pass filter}
         \PY{n}{diff\PYZus{}band}\PY{o}{=}\PY{n}{cv2}\PY{o}{.}\PY{n}{imread}\PY{p}{(}\PY{l+s+s1}{\PYZsq{}}\PY{l+s+s1}{diff\PYZus{}band.jpg}\PY{l+s+s1}{\PYZsq{}}\PY{p}{)}
         \PY{n}{dct\PYZus{}img}\PY{o}{=}\PY{n}{dct2}\PY{p}{(}\PY{n}{img}\PY{p}{)}
         \PY{n}{diff\PYZus{}band}\PY{o}{=}\PY{n}{cv2}\PY{o}{.}\PY{n}{cvtColor}\PY{p}{(}\PY{n}{diff\PYZus{}band}\PY{p}{,}\PY{n}{cv2}\PY{o}{.}\PY{n}{COLOR\PYZus{}BGR2RGB}\PY{p}{)}
         \PY{n}{plt}\PY{o}{.}\PY{n}{subplot}\PY{p}{(}\PY{l+m+mi}{1}\PY{p}{,}\PY{l+m+mi}{3}\PY{p}{,}\PY{l+m+mi}{3}\PY{p}{)}
         \PY{n}{plt}\PY{o}{.}\PY{n}{imshow}\PY{p}{(}\PY{n}{diff\PYZus{}band}\PY{p}{)}
         \PY{n}{plt}\PY{o}{.}\PY{n}{title}\PY{p}{(}\PY{l+s+s1}{\PYZsq{}}\PY{l+s+s1}{difference}\PY{l+s+s1}{\PYZsq{}}\PY{p}{)}
         \PY{n}{plt}\PY{o}{.}\PY{n}{show}\PY{p}{(}\PY{p}{)}
\end{Verbatim}


    \begin{center}
    \adjustimage{max size={0.9\linewidth}{0.9\paperheight}}{output_26_0.png}
    \end{center}
    { \hspace*{\fill} \\}
    
    \hypertarget{conclusion-and-analysis}{%
\section{Conclusion and Analysis}\label{conclusion-and-analysis}}

    The goal applying low-pass and high-pass filter to the images is
achieved by doing 2-D DCT, zeroing certain zones of the matrix and then
applying iDCT to the image data.

The results obviously express that the most of the information
describing the images exist in low frequency. As shown in these two
examples, the image processed by the low-pass filter whose dimensions
are only 10x10 shows more blur than that applied the low-pass filter
with 40x40 dimensions. Supprisingly, even only 10x10 dimensions are
applied, the processed image still look not bad and can be identified
easily compared to the one applied by the high-pass filter. And the
high-pass filter can be used for edge detection since it keeps more
information on the edges. Moreover, since human eyes have limitations on
high frequency, some modifications such as removing the data from high
frequency are not recognized by people. Therefore, image compression to
reduce dimensions or decrease the storage space is practical and
meaningful.


    % Add a bibliography block to the postdoc
    
    
    
    \end{document}
